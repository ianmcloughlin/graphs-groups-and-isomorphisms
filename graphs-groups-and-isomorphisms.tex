\documentclass{iansnotes}

\title{Graphs, Groups, and Isomorphisms}
\author{ian.mcloughlin@atu.ie}
\date{\today}

\begin{document}

\maketitle

Are all programming languages equally capable?
What are the limits of computation?
How can we discuss computation without worrying about the details of specific machines?
To answer these questions, we need a simple way of describing problems.
In the following article we will define the basic building blocks of computing.


\section{References}
  We will use two reference texts: Norman Biggs' \emph{Discrete Mathematics}\autocite{biggs} and Michael Sipser's \emph{Introduction to the Theory of Computation}\autocite{sipser}.
  Another good resource is the Open Logic Text~\autocite{openlogictext}.
  On the practical side, I recommend also use The Python Tutorial\autocite{pythontutorial}, The Python Software Foundation's official tutorial for Python.


\section{Graphs} 
  A graph is
  \begin{minted}{python}
def f(x):
  ans = 1
  for i in range(x):
    ans = ans * i
  return ans
  \end{minted}
  In the earlier definition of a function we made no reference to algorithms.
  To mimic those kinds of functions we can just use a dictionary.
  \begin{minted}{python}
def f(x):
  pairs = {1: 1, 2: 4, 3: 9}
  return pairs[x]
  \end{minted}

\section{Further Topics}
  Now that we have defined our basic building blocks, we can discuss topics in computation and computability.
\end{document} 